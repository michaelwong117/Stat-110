\documentclass[10pt]{article}
 
\usepackage[margin=1in]{geometry} 
\usepackage{amsmath,amsthm,amssymb, graphicx, multicol, array}
\usepackage[parfill]{parskip}
\newcommand{\N}{\mathbb{N}}
\newcommand{\Z}{\mathbb{Z}}
 
\newenvironment{problem}[2][Problem]{\begin{trivlist}
\item[\hskip \labelsep {\bfseries #1}\hskip \labelsep {\bfseries #2.}]}{\end{trivlist}}

\begin{document}
 
\title{Strategic Practice 1, Fall 2011}
\author{Michael Wong\\
Stat 110, Harvard University}
\maketitle

\section{Naive Definition of Probability}


\begin{problem}{2} 
A random 5 card poker hand is dealt from a standard deck of cards. Find the probability of each of the following (in terms of binomial coefficients). WRONG SOLUTION. 
\end{problem}

\begin{question} 

\textbf{(b)} Two pair (e.g., two 3's, two 7's, and an Ace)

\end{question}

\begin{proof}[Solution]
\[
       \frac{13 \times \binom{4}{2} \times 12 \times \binom{4}{2} \times \binom{48}{1}}{\binom{52}{5}}
\]
\end{proof}

\begin{proof}[Explanation]

There are 52 cards total, and order does not matter for card hands, so there are again a total of \(\binom{52}{5}\) possible five-card hands. There are 13 unique types of cards, and 4 of each (e.g., 2, 2, 2, 2) that are possible to pair. We first need to For the first pair, we need to choose a card that we are pairing (13 cards available) and then we need to choose 2 cards out of 4: \(\binom{4}{2}\), so we have $13 \times \binom{4}{2}$. For the second pair, we need to again choose a card that we are pairing (13 - the 1st card = 12 cards available) and then we need to choose 2 cards out of 4: \(\binom{4}{2}\), so we have $12 \times \binom{4}{2}$. For the remaining card, there are 48 cards left and we need to choose 1, so we have $\binom{48}{1}$. Thus, our total probability is
\[
       \frac{13 \times \binom{4}{2} \times 12 \times \binom{4}{2} \times \binom{48}{1}}{\binom{52}{5}}
\]

\end{proof}


\end{document}