\documentclass[10pt]{article}
 
\usepackage[margin=1in]{geometry} 
\usepackage{amsmath,amsthm,amssymb, graphicx, multicol, array}
 
\newcommand{\N}{\mathbb{N}}
\newcommand{\Z}{\mathbb{Z}}
 
\newenvironment{problem}[2][Problem]{\begin{trivlist}
\item[\hskip \labelsep {\bfseries #1}\hskip \labelsep {\bfseries #2.}]}{\end{trivlist}}

\begin{document}
 
\title{Strategic Practice 1, Fall 2011}
\author{Michael Wong\\
Stat 110, Harvard University}
\maketitle
 
\begin{problem}{1}
(a) (probability that the total after rolling 4 fair dice is 21) \underline{\hspace{1cm}}
(probability that the total after rolling 4 fair dice is 22) NOT PRECISE SOLUTION. 
\end{problem}

\begin{proof}[Solution]

(probability that the total after rolling 4 fair dice is 21) $>$
(probability that the total after rolling 4 fair dice is 22)
\end{proof}

\begin{proof}[Explanation]

The expected value of the total after rolling one fair die is $21 / 6 = 3.5$. The expected value of the total after rolling four fair dice is thus $3.5 \times 4 = 14$. Since 21 is closer to the expected value than 22, it is the more probable roll.
\end{proof}

\end{document}
